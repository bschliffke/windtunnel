%% Generated by Sphinx.
\def\sphinxdocclass{report}
\documentclass[letterpaper,10pt,english]{sphinxmanual}
\ifdefined\pdfpxdimen
   \let\sphinxpxdimen\pdfpxdimen\else\newdimen\sphinxpxdimen
\fi \sphinxpxdimen=.75bp\relax

\PassOptionsToPackage{warn}{textcomp}
\usepackage[utf8]{inputenc}
\ifdefined\DeclareUnicodeCharacter
 \ifdefined\DeclareUnicodeCharacterAsOptional
  \DeclareUnicodeCharacter{"00A0}{\nobreakspace}
  \DeclareUnicodeCharacter{"2500}{\sphinxunichar{2500}}
  \DeclareUnicodeCharacter{"2502}{\sphinxunichar{2502}}
  \DeclareUnicodeCharacter{"2514}{\sphinxunichar{2514}}
  \DeclareUnicodeCharacter{"251C}{\sphinxunichar{251C}}
  \DeclareUnicodeCharacter{"2572}{\textbackslash}
 \else
  \DeclareUnicodeCharacter{00A0}{\nobreakspace}
  \DeclareUnicodeCharacter{2500}{\sphinxunichar{2500}}
  \DeclareUnicodeCharacter{2502}{\sphinxunichar{2502}}
  \DeclareUnicodeCharacter{2514}{\sphinxunichar{2514}}
  \DeclareUnicodeCharacter{251C}{\sphinxunichar{251C}}
  \DeclareUnicodeCharacter{2572}{\textbackslash}
 \fi
\fi
\usepackage{cmap}
\usepackage[T1]{fontenc}
\usepackage{amsmath,amssymb,amstext}
\usepackage{babel}
\usepackage{times}
\usepackage[Bjarne]{fncychap}
\usepackage{sphinx}

\usepackage{geometry}

% Include hyperref last.
\usepackage{hyperref}
% Fix anchor placement for figures with captions.
\usepackage{hypcap}% it must be loaded after hyperref.
% Set up styles of URL: it should be placed after hyperref.
\urlstyle{same}

\addto\captionsenglish{\renewcommand{\figurename}{Fig.}}
\addto\captionsenglish{\renewcommand{\tablename}{Table}}
\addto\captionsenglish{\renewcommand{\literalblockname}{Listing}}

\addto\captionsenglish{\renewcommand{\literalblockcontinuedname}{continued from previous page}}
\addto\captionsenglish{\renewcommand{\literalblockcontinuesname}{continues on next page}}

\addto\extrasenglish{\def\pageautorefname{page}}

\setcounter{tocdepth}{1}



\title{windtunnel Documentation}
\date{Aug 08, 2018}
\release{0.0.1}
\author{Benyamin Schliffke \& Jessica Wiedemeier}
\newcommand{\sphinxlogo}{\vbox{}}
\renewcommand{\releasename}{Release}
\makeindex

\begin{document}

\maketitle
\sphinxtableofcontents
\phantomsection\label{\detokenize{index::doc}}


A collection of tools for basic boundary layer and concentration measurements analysis with Python 3.


\chapter{windtunnel}
\label{\detokenize{index:module-windtunnel}}\label{\detokenize{index:windtunnel}}\label{\detokenize{index:welcome-to-windtunnel-s-documentation}}\index{windtunnel (module)}
Python package for basic boundary layer and concentration measurement analysis.
\index{PointConcentration (class in windtunnel)}

\begin{fulllineitems}
\phantomsection\label{\detokenize{index:windtunnel.PointConcentration}}\pysiglinewithargsret{\sphinxbfcode{\sphinxupquote{class }}\sphinxcode{\sphinxupquote{windtunnel.}}\sphinxbfcode{\sphinxupquote{PointConcentration}}}{\emph{time}, \emph{wtref}, \emph{slow\_FID}, \emph{fast\_FID}, \emph{open\_rate}}{}
PointConcentration is a class that holds data collected during
a continuous release point concentration measurement. The class can hold
the raw time series, the corresponding wtref and all other quantities
necessary to analyse the time series. All the information in a
PointConcentration object can be saved to a txt file.
@parameter: time, type = np.array
@parameter: wtref, type = np.array
@parameter: fast\_FID, type = np.array
@parameter: slow\_FID, type = np.array
@parameter: open\_rate, type = np.array
\index{ambient\_conditions() (windtunnel.PointConcentration method)}

\begin{fulllineitems}
\phantomsection\label{\detokenize{index:windtunnel.PointConcentration.ambient_conditions}}\pysiglinewithargsret{\sphinxbfcode{\sphinxupquote{ambient\_conditions}}}{\emph{x}, \emph{y}, \emph{z}, \emph{pressure}, \emph{temperature}, \emph{calibration\_curve}, \emph{mass\_flow\_controller}, \emph{calibration\_factor=0}}{}
Collect ambient conditions during measurement. pressure in {[}Pa{]},
temperature in {[}°C{]}.

\end{fulllineitems}

\index{calc\_c\_star() (windtunnel.PointConcentration method)}

\begin{fulllineitems}
\phantomsection\label{\detokenize{index:windtunnel.PointConcentration.calc_c_star}}\pysiglinewithargsret{\sphinxbfcode{\sphinxupquote{calc\_c\_star}}}{}{}
Calculate dimensionless concentration. {[}-{]}

\end{fulllineitems}

\index{calc\_full\_scale\_concentration() (windtunnel.PointConcentration method)}

\begin{fulllineitems}
\phantomsection\label{\detokenize{index:windtunnel.PointConcentration.calc_full_scale_concentration}}\pysiglinewithargsret{\sphinxbfcode{\sphinxupquote{calc\_full\_scale\_concentration}}}{}{}
Calculate full scale concentration in {[}ppmV{]}.

\end{fulllineitems}

\index{calc\_full\_scale\_flow\_rate() (windtunnel.PointConcentration method)}

\begin{fulllineitems}
\phantomsection\label{\detokenize{index:windtunnel.PointConcentration.calc_full_scale_flow_rate}}\pysiglinewithargsret{\sphinxbfcode{\sphinxupquote{calc\_full\_scale\_flow\_rate}}}{}{}
Convert flow rate to full scale flow rate in {[}m\textasciicircum{}3/s{]}.

\end{fulllineitems}

\index{calc\_full\_scale\_time() (windtunnel.PointConcentration method)}

\begin{fulllineitems}
\phantomsection\label{\detokenize{index:windtunnel.PointConcentration.calc_full_scale_time}}\pysiglinewithargsret{\sphinxbfcode{\sphinxupquote{calc\_full\_scale\_time}}}{}{}
Calculate full scale timesteps in {[}s{]}.

\end{fulllineitems}

\index{calc\_model\_mass\_flow\_rate() (windtunnel.PointConcentration method)}

\begin{fulllineitems}
\phantomsection\label{\detokenize{index:windtunnel.PointConcentration.calc_model_mass_flow_rate}}\pysiglinewithargsret{\sphinxbfcode{\sphinxupquote{calc\_model\_mass\_flow\_rate}}}{}{}
Calculate the model scale flow rate in {[}kg/s{]}.

\end{fulllineitems}

\index{calc\_net\_concentration() (windtunnel.PointConcentration method)}

\begin{fulllineitems}
\phantomsection\label{\detokenize{index:windtunnel.PointConcentration.calc_net_concentration}}\pysiglinewithargsret{\sphinxbfcode{\sphinxupquote{calc\_net\_concentration}}}{}{}
Calculate net concentration in {[}ppmV{]}.

\end{fulllineitems}

\index{calc\_wtref\_mean() (windtunnel.PointConcentration method)}

\begin{fulllineitems}
\phantomsection\label{\detokenize{index:windtunnel.PointConcentration.calc_wtref_mean}}\pysiglinewithargsret{\sphinxbfcode{\sphinxupquote{calc\_wtref\_mean}}}{}{}
Calculate scaled wtref mean in {[}m/s{]}.

\end{fulllineitems}

\index{clear\_zeros() (windtunnel.PointConcentration method)}

\begin{fulllineitems}
\phantomsection\label{\detokenize{index:windtunnel.PointConcentration.clear_zeros}}\pysiglinewithargsret{\sphinxbfcode{\sphinxupquote{clear\_zeros}}}{}{}
Clear and count zeros in concentration measurements.

\end{fulllineitems}

\index{convert\_temperature() (windtunnel.PointConcentration method)}

\begin{fulllineitems}
\phantomsection\label{\detokenize{index:windtunnel.PointConcentration.convert_temperature}}\pysiglinewithargsret{\sphinxbfcode{\sphinxupquote{convert\_temperature}}}{}{}
Convert ambient temperature to °K.

\end{fulllineitems}

\index{from\_file() (windtunnel.PointConcentration class method)}

\begin{fulllineitems}
\phantomsection\label{\detokenize{index:windtunnel.PointConcentration.from_file}}\pysiglinewithargsret{\sphinxbfcode{\sphinxupquote{classmethod }}\sphinxbfcode{\sphinxupquote{from\_file}}}{\emph{filename}}{}
Create PointConcentration object from file. open\_rate is converted
to \%.

\end{fulllineitems}

\index{full\_scale\_information() (windtunnel.PointConcentration method)}

\begin{fulllineitems}
\phantomsection\label{\detokenize{index:windtunnel.PointConcentration.full_scale_information}}\pysiglinewithargsret{\sphinxbfcode{\sphinxupquote{full\_scale\_information}}}{\emph{full\_scale\_wtref}, \emph{full\_scale\_flow\_rate}}{}
Collect information on desired full scale information.
full\_scale\_wtref in {[}m/s{]}. full\_scale\_flow\_rate is automatically
adjusted to standard atmosphere conditions.
input in {[}kg/s{]}, output in {[}m\textasciicircum{}3/s{]}.

\end{fulllineitems}

\index{save2file\_avg() (windtunnel.PointConcentration method)}

\begin{fulllineitems}
\phantomsection\label{\detokenize{index:windtunnel.PointConcentration.save2file_avg}}\pysiglinewithargsret{\sphinxbfcode{\sphinxupquote{save2file\_avg}}}{\emph{filename}, \emph{out\_dir=None}}{}
Save average full scale and model scale data from
PointConcentration object to txt file. filename must include ‘.txt’
ending. If no out\_dir directory is provided ‘./’ is set as standard.
@parameter: filename, type = str
@parameter: out\_dir, type = str

\end{fulllineitems}

\index{save2file\_fs() (windtunnel.PointConcentration method)}

\begin{fulllineitems}
\phantomsection\label{\detokenize{index:windtunnel.PointConcentration.save2file_fs}}\pysiglinewithargsret{\sphinxbfcode{\sphinxupquote{save2file\_fs}}}{\emph{filename}, \emph{out\_dir=None}}{}
Save full scale and model scale data from PointConcentration object
to txt file. filename must include ‘.txt’ ending. If no out\_dir
directory is provided ‘./’ is set as standard.
@parameter: filename, type = str
@parameter: out\_dir, type = str

\end{fulllineitems}

\index{save2file\_ms() (windtunnel.PointConcentration method)}

\begin{fulllineitems}
\phantomsection\label{\detokenize{index:windtunnel.PointConcentration.save2file_ms}}\pysiglinewithargsret{\sphinxbfcode{\sphinxupquote{save2file\_ms}}}{\emph{filename}, \emph{out\_dir=None}}{}
Save model scale data from PointConcentration object to txt file.
filename must include ‘.txt’ ending. If no out\_dir directory is
provided ‘./’ is set as standard.
@parameter: filename, type = str
@parameter: out\_dir, type = str

\end{fulllineitems}

\index{scaling\_information() (windtunnel.PointConcentration method)}

\begin{fulllineitems}
\phantomsection\label{\detokenize{index:windtunnel.PointConcentration.scaling_information}}\pysiglinewithargsret{\sphinxbfcode{\sphinxupquote{scaling\_information}}}{\emph{scaling\_factor}, \emph{scale}, \emph{ref\_length}, \emph{ref\_height}}{}
Collect data necessary to scale the results. unit: {[}m{]}, where
applicable.

\end{fulllineitems}

\index{to\_full\_scale() (windtunnel.PointConcentration method)}

\begin{fulllineitems}
\phantomsection\label{\detokenize{index:windtunnel.PointConcentration.to_full_scale}}\pysiglinewithargsret{\sphinxbfcode{\sphinxupquote{to\_full\_scale}}}{}{}
Return all quantities to full scale. Requires XXXXXX to be
specified.

\end{fulllineitems}

\index{tracer\_information() (windtunnel.PointConcentration method)}

\begin{fulllineitems}
\phantomsection\label{\detokenize{index:windtunnel.PointConcentration.tracer_information}}\pysiglinewithargsret{\sphinxbfcode{\sphinxupquote{tracer\_information}}}{\emph{gas\_name}, \emph{mol\_weight}, \emph{gas\_factor}}{}
Collect information on tracer gas used during measurement.
Molecular weight in {[}kg/mol{]}.

\end{fulllineitems}


\end{fulllineitems}

\index{PuffConcentration (class in windtunnel)}

\begin{fulllineitems}
\phantomsection\label{\detokenize{index:windtunnel.PuffConcentration}}\pysiglinewithargsret{\sphinxbfcode{\sphinxupquote{class }}\sphinxcode{\sphinxupquote{windtunnel.}}\sphinxbfcode{\sphinxupquote{PuffConcentration}}}{\emph{time}, \emph{wtref}, \emph{slow\_FID}, \emph{fast\_FID}, \emph{signal}, \emph{open\_rate}}{}
PuffConcentration is a class that holds data collected during 
a puff release point concentration measurement. The class can hold
the raw time series, the corresponding wtref and all other quantities
necessary to analyse the time series. The PuffConcentration class
inherits from pandas.DataFrame, thus offers all of the functionality
offered by pandas (e.g. DataFrame.plot.hist(), DataFrame.to\_excel(),
or DataFrame.rolling().mean()) All the information in a
PuffConcentration object can be saved to a txt file, as well as all
file type offered by pandas.
@parameter: time, type = pd.Series
@parameter: wtref, type = np.array
@parameter: fast\_FID, type = pd.Series
@parameter: slow\_FID, type = pd.Series
@parameter: signal, type = np.array
@parameter: open\_rate, type = np.array
\index{apply\_threshold\_concentration() (windtunnel.PuffConcentration method)}

\begin{fulllineitems}
\phantomsection\label{\detokenize{index:windtunnel.PuffConcentration.apply_threshold_concentration}}\pysiglinewithargsret{\sphinxbfcode{\sphinxupquote{apply\_threshold\_concentration}}}{\emph{threshold\_concentration=0.0}}{}
Apply a given threshold concentration to peak\_concentration to 
remove weak puffs. The default value for threshold\_concentration 
is 0. (float).

\end{fulllineitems}

\index{avg\_arrival\_time (windtunnel.PuffConcentration attribute)}

\begin{fulllineitems}
\phantomsection\label{\detokenize{index:windtunnel.PuffConcentration.avg_arrival_time}}\pysigline{\sphinxbfcode{\sphinxupquote{avg\_arrival\_time}}}
Get average arrival time.

\end{fulllineitems}

\index{avg\_ascent\_time (windtunnel.PuffConcentration attribute)}

\begin{fulllineitems}
\phantomsection\label{\detokenize{index:windtunnel.PuffConcentration.avg_ascent_time}}\pysigline{\sphinxbfcode{\sphinxupquote{avg\_ascent\_time}}}
Get average ascent time.

\end{fulllineitems}

\index{avg\_descent\_time (windtunnel.PuffConcentration attribute)}

\begin{fulllineitems}
\phantomsection\label{\detokenize{index:windtunnel.PuffConcentration.avg_descent_time}}\pysigline{\sphinxbfcode{\sphinxupquote{avg\_descent\_time}}}
Get average descent time.

\end{fulllineitems}

\index{avg\_leaving\_time (windtunnel.PuffConcentration attribute)}

\begin{fulllineitems}
\phantomsection\label{\detokenize{index:windtunnel.PuffConcentration.avg_leaving_time}}\pysigline{\sphinxbfcode{\sphinxupquote{avg\_leaving\_time}}}
Get average leaving time.

\end{fulllineitems}

\index{avg\_peak\_concentration (windtunnel.PuffConcentration attribute)}

\begin{fulllineitems}
\phantomsection\label{\detokenize{index:windtunnel.PuffConcentration.avg_peak_concentration}}\pysigline{\sphinxbfcode{\sphinxupquote{avg\_peak\_concentration}}}
Get average peak concentration.

\end{fulllineitems}

\index{avg\_peak\_time (windtunnel.PuffConcentration attribute)}

\begin{fulllineitems}
\phantomsection\label{\detokenize{index:windtunnel.PuffConcentration.avg_peak_time}}\pysigline{\sphinxbfcode{\sphinxupquote{avg\_peak\_time}}}
Get average peak time.

\end{fulllineitems}

\index{calc\_net\_concentration() (windtunnel.PuffConcentration method)}

\begin{fulllineitems}
\phantomsection\label{\detokenize{index:windtunnel.PuffConcentration.calc_net_concentration}}\pysiglinewithargsret{\sphinxbfcode{\sphinxupquote{calc\_net\_concentration}}}{}{}
Calculate net concentration in {[}ppmV{]}.

\end{fulllineitems}

\index{calc\_release\_length() (windtunnel.PuffConcentration method)}

\begin{fulllineitems}
\phantomsection\label{\detokenize{index:windtunnel.PuffConcentration.calc_release_length}}\pysiglinewithargsret{\sphinxbfcode{\sphinxupquote{calc\_release\_length}}}{}{}
Calculate the length of each release period. Returns an np.array
containing the duration of each release period.

\end{fulllineitems}

\index{check\_against\_avg\_puff() (windtunnel.PuffConcentration method)}

\begin{fulllineitems}
\phantomsection\label{\detokenize{index:windtunnel.PuffConcentration.check_against_avg_puff}}\pysiglinewithargsret{\sphinxbfcode{\sphinxupquote{check\_against\_avg\_puff}}}{}{}
Check each puff against the average puff of the time series.

\end{fulllineitems}

\index{detect\_arrival\_time() (windtunnel.PuffConcentration method)}

\begin{fulllineitems}
\phantomsection\label{\detokenize{index:windtunnel.PuffConcentration.detect_arrival_time}}\pysiglinewithargsret{\sphinxbfcode{\sphinxupquote{detect\_arrival\_time}}}{}{}
Detects the beginning of each puff. Returns an np.array 
containing the first timestamp of each puff.

\end{fulllineitems}

\index{detect\_begin\_release\_index() (windtunnel.PuffConcentration method)}

\begin{fulllineitems}
\phantomsection\label{\detokenize{index:windtunnel.PuffConcentration.detect_begin_release_index}}\pysiglinewithargsret{\sphinxbfcode{\sphinxupquote{detect\_begin\_release\_index}}}{}{}
Detects the indices of the end of each release period. Returns a
list containing the index of the last timestamp of each release 
period.

\end{fulllineitems}

\index{detect\_begin\_release\_period() (windtunnel.PuffConcentration method)}

\begin{fulllineitems}
\phantomsection\label{\detokenize{index:windtunnel.PuffConcentration.detect_begin_release_period}}\pysiglinewithargsret{\sphinxbfcode{\sphinxupquote{detect\_begin\_release\_period}}}{}{}
Detects the beginning of each release period. Returns an np.array 
containing the first timestamp of each release period.

\end{fulllineitems}

\index{detect\_end\_release\_index() (windtunnel.PuffConcentration method)}

\begin{fulllineitems}
\phantomsection\label{\detokenize{index:windtunnel.PuffConcentration.detect_end_release_index}}\pysiglinewithargsret{\sphinxbfcode{\sphinxupquote{detect\_end\_release\_index}}}{}{}
Detects the indices of the end of each release period. Returns a
list containing the index of the last timestamp of each release 
period.

\end{fulllineitems}

\index{detect\_end\_release\_period() (windtunnel.PuffConcentration method)}

\begin{fulllineitems}
\phantomsection\label{\detokenize{index:windtunnel.PuffConcentration.detect_end_release_period}}\pysiglinewithargsret{\sphinxbfcode{\sphinxupquote{detect\_end\_release\_period}}}{}{}
Detects the end of each release period. Returns an np.array 
containing the last timestamp of each release period.

\end{fulllineitems}

\index{detect\_leaving\_time() (windtunnel.PuffConcentration method)}

\begin{fulllineitems}
\phantomsection\label{\detokenize{index:windtunnel.PuffConcentration.detect_leaving_time}}\pysiglinewithargsret{\sphinxbfcode{\sphinxupquote{detect\_leaving\_time}}}{}{}
Detects the end of each puff. Returns an np.array 
containing the last timestamp of each puff.

\end{fulllineitems}

\index{from\_file() (windtunnel.PuffConcentration class method)}

\begin{fulllineitems}
\phantomsection\label{\detokenize{index:windtunnel.PuffConcentration.from_file}}\pysiglinewithargsret{\sphinxbfcode{\sphinxupquote{classmethod }}\sphinxbfcode{\sphinxupquote{from\_file}}}{\emph{filename}}{}
Create PuffConcentration object from file. open\_rate is converted
to \%.
:type filename: str

\end{fulllineitems}

\index{get\_ascent\_time() (windtunnel.PuffConcentration method)}

\begin{fulllineitems}
\phantomsection\label{\detokenize{index:windtunnel.PuffConcentration.get_ascent_time}}\pysiglinewithargsret{\sphinxbfcode{\sphinxupquote{get\_ascent\_time}}}{}{}
Calculate the ascent time between arrrival time and peak time. 
Returns an np.array.

\end{fulllineitems}

\index{get\_descent\_time() (windtunnel.PuffConcentration method)}

\begin{fulllineitems}
\phantomsection\label{\detokenize{index:windtunnel.PuffConcentration.get_descent_time}}\pysiglinewithargsret{\sphinxbfcode{\sphinxupquote{get\_descent\_time}}}{}{}
Calculate the ascent time between arrrival time and peak time. 
Returns an np.array.

\end{fulllineitems}

\index{get\_dosage() (windtunnel.PuffConcentration method)}

\begin{fulllineitems}
\phantomsection\label{\detokenize{index:windtunnel.PuffConcentration.get_dosage}}\pysiglinewithargsret{\sphinxbfcode{\sphinxupquote{get\_dosage}}}{}{}
Calculates the dosage of each puff between two release times.

\end{fulllineitems}

\index{get\_peak\_concentration() (windtunnel.PuffConcentration method)}

\begin{fulllineitems}
\phantomsection\label{\detokenize{index:windtunnel.PuffConcentration.get_peak_concentration}}\pysiglinewithargsret{\sphinxbfcode{\sphinxupquote{get\_peak\_concentration}}}{}{}
Acquire peak concentration of each puff. Returns a list.

\end{fulllineitems}

\index{get\_peak\_time() (windtunnel.PuffConcentration method)}

\begin{fulllineitems}
\phantomsection\label{\detokenize{index:windtunnel.PuffConcentration.get_peak_time}}\pysiglinewithargsret{\sphinxbfcode{\sphinxupquote{get\_peak\_time}}}{}{}
Acquire peak time of each puff. Returns a list.

\end{fulllineitems}

\index{get\_puff\_statistics() (windtunnel.PuffConcentration method)}

\begin{fulllineitems}
\phantomsection\label{\detokenize{index:windtunnel.PuffConcentration.get_puff_statistics}}\pysiglinewithargsret{\sphinxbfcode{\sphinxupquote{get\_puff\_statistics}}}{}{}
Returns DataFrame with all puff information.

\end{fulllineitems}

\index{get\_residence\_time() (windtunnel.PuffConcentration method)}

\begin{fulllineitems}
\phantomsection\label{\detokenize{index:windtunnel.PuffConcentration.get_residence_time}}\pysiglinewithargsret{\sphinxbfcode{\sphinxupquote{get\_residence\_time}}}{}{}
Calculate the residence time of each puff. Returns an np.array.

\end{fulllineitems}

\index{max\_puffs (windtunnel.PuffConcentration attribute)}

\begin{fulllineitems}
\phantomsection\label{\detokenize{index:windtunnel.PuffConcentration.max_puffs}}\pysigline{\sphinxbfcode{\sphinxupquote{max\_puffs}}}
Get maximum number of puffs. Deduced from the length of 
release\_length.

\end{fulllineitems}

\index{offset\_correction() (windtunnel.PuffConcentration method)}

\begin{fulllineitems}
\phantomsection\label{\detokenize{index:windtunnel.PuffConcentration.offset_correction}}\pysiglinewithargsret{\sphinxbfcode{\sphinxupquote{offset\_correction}}}{}{}
Correct a non-zero offset in the concentration measured.

\end{fulllineitems}

\index{save2file() (windtunnel.PuffConcentration method)}

\begin{fulllineitems}
\phantomsection\label{\detokenize{index:windtunnel.PuffConcentration.save2file}}\pysiglinewithargsret{\sphinxbfcode{\sphinxupquote{save2file}}}{\emph{filename}, \emph{out\_dir=None}}{}
Save data from PointConcentration object to txt file. filename must
include ‘.txt’ ending. If no out\_dir directory is provided ‘./’ is set 
as standard.
@parameter: filename, type = str
@parameter: out\_dir, type = str

\end{fulllineitems}


\end{fulllineitems}

\index{Timeseries (class in windtunnel)}

\begin{fulllineitems}
\phantomsection\label{\detokenize{index:windtunnel.Timeseries}}\pysiglinewithargsret{\sphinxbfcode{\sphinxupquote{class }}\sphinxcode{\sphinxupquote{windtunnel.}}\sphinxbfcode{\sphinxupquote{Timeseries}}}{\emph{u}, \emph{v}, \emph{x=None}, \emph{y=None}, \emph{z=None}, \emph{t\_arr=None}, \emph{t\_transit=None}, \emph{tau=10000}}{}
Timeseries is a class that holds data collected by the BSA software in
the standard BSA software output. The class can hold die raw timeseries,
the corresponding wtref, the components and coordinates of each
measurement as well as the mean wind magnitude and the mean wind direction.
The raw timeseries can be processed by nondimensionalising it, adapting the
scale, making it equidistant and masking outliers. All the information in
a Timeseries object can be saved to a txt file.
@parameter: u, type = np.array
@parameter: v, type = np.array
@parameter: x, type = float
@parameter: y, type = float
@parameter: z, type = float
@parameter: t\_arr, type = np.array
@parameter: t\_transit, type = np.array
@parameter: tau, type = int or float - time scale in milliseconds
\index{adapt\_scale() (windtunnel.Timeseries method)}

\begin{fulllineitems}
\phantomsection\label{\detokenize{index:windtunnel.Timeseries.adapt_scale}}\pysiglinewithargsret{\sphinxbfcode{\sphinxupquote{adapt\_scale}}}{\emph{scale}}{}
Convert timeseries from model scale to full scale.
@parameter: scale, type = float

\end{fulllineitems}

\index{calc\_direction() (windtunnel.Timeseries method)}

\begin{fulllineitems}
\phantomsection\label{\detokenize{index:windtunnel.Timeseries.calc_direction}}\pysiglinewithargsret{\sphinxbfcode{\sphinxupquote{calc\_direction}}}{}{}
Calculate wind direction from components.

\end{fulllineitems}

\index{calc\_equidistant\_timesteps() (windtunnel.Timeseries method)}

\begin{fulllineitems}
\phantomsection\label{\detokenize{index:windtunnel.Timeseries.calc_equidistant_timesteps}}\pysiglinewithargsret{\sphinxbfcode{\sphinxupquote{calc\_equidistant\_timesteps}}}{}{}
Create equidistant time series.

\end{fulllineitems}

\index{calc\_magnitude() (windtunnel.Timeseries method)}

\begin{fulllineitems}
\phantomsection\label{\detokenize{index:windtunnel.Timeseries.calc_magnitude}}\pysiglinewithargsret{\sphinxbfcode{\sphinxupquote{calc\_magnitude}}}{}{}
Calculate wind magnitude from components.

\end{fulllineitems}

\index{calc\_perturbations() (windtunnel.Timeseries method)}

\begin{fulllineitems}
\phantomsection\label{\detokenize{index:windtunnel.Timeseries.calc_perturbations}}\pysiglinewithargsret{\sphinxbfcode{\sphinxupquote{calc\_perturbations}}}{}{}
Calculates u’ and v’ relative to the mean of each tau-long data 
segment

\end{fulllineitems}

\index{from\_file() (windtunnel.Timeseries class method)}

\begin{fulllineitems}
\phantomsection\label{\detokenize{index:windtunnel.Timeseries.from_file}}\pysiglinewithargsret{\sphinxbfcode{\sphinxupquote{classmethod }}\sphinxbfcode{\sphinxupquote{from\_file}}}{\emph{filename}}{}
Create Timeseries object from file.

\end{fulllineitems}

\index{get\_wind\_comps() (windtunnel.Timeseries method)}

\begin{fulllineitems}
\phantomsection\label{\detokenize{index:windtunnel.Timeseries.get_wind_comps}}\pysiglinewithargsret{\sphinxbfcode{\sphinxupquote{get\_wind\_comps}}}{\emph{filename}}{}
Get wind components from filename.
@parameter: filename, type = str

\end{fulllineitems}

\index{get\_wtref() (windtunnel.Timeseries method)}

\begin{fulllineitems}
\phantomsection\label{\detokenize{index:windtunnel.Timeseries.get_wtref}}\pysiglinewithargsret{\sphinxbfcode{\sphinxupquote{get\_wtref}}}{\emph{wtref\_path}, \emph{filename}, \emph{index=0}, \emph{vscale=1.0}}{}
Reads wtref-file selected by the time series name ‘filename’ and
scales wtref with vscale. vscale is set to 1 as standard. index
accesses only the one wtref value that is associated to the current
file.
@parameter: path, type = string
@parameter: filename, type = string
@parameter: index, type = int
@parameter: vscale, type = float

\end{fulllineitems}

\index{mask\_outliers() (windtunnel.Timeseries method)}

\begin{fulllineitems}
\phantomsection\label{\detokenize{index:windtunnel.Timeseries.mask_outliers}}\pysiglinewithargsret{\sphinxbfcode{\sphinxupquote{mask\_outliers}}}{\emph{std\_mask=5.0}}{}
Mask outliers and print number of outliers. std\_mask specifies the
threshold for a value to be considered an outlier. 5 is the default
value for std\_mask.
@parameter: std\_mask, type = float

\end{fulllineitems}

\index{mean\_direction (windtunnel.Timeseries attribute)}

\begin{fulllineitems}
\phantomsection\label{\detokenize{index:windtunnel.Timeseries.mean_direction}}\pysigline{\sphinxbfcode{\sphinxupquote{mean\_direction}}}
Calculate mean wind direction from components relative to the wind
tunnels axis.

\end{fulllineitems}

\index{mean\_magnitude (windtunnel.Timeseries attribute)}

\begin{fulllineitems}
\phantomsection\label{\detokenize{index:windtunnel.Timeseries.mean_magnitude}}\pysigline{\sphinxbfcode{\sphinxupquote{mean\_magnitude}}}
Calculate mean wind magnitude from unweighted components.

\end{fulllineitems}

\index{nondimensionalise() (windtunnel.Timeseries method)}

\begin{fulllineitems}
\phantomsection\label{\detokenize{index:windtunnel.Timeseries.nondimensionalise}}\pysiglinewithargsret{\sphinxbfcode{\sphinxupquote{nondimensionalise}}}{}{}
Nondimensionalise the data. wtref is set to 1 if no wtref is
speciefied.

\end{fulllineitems}

\index{save2file() (windtunnel.Timeseries method)}

\begin{fulllineitems}
\phantomsection\label{\detokenize{index:windtunnel.Timeseries.save2file}}\pysiglinewithargsret{\sphinxbfcode{\sphinxupquote{save2file}}}{\emph{filename}, \emph{out\_dir=None}}{}
Save data from Timeseries object to txt file. filename must include
‘.txt’ ending. If no out\_dir directory is provided
‘C:/Users/{[}your\_u\_number{]}/Desktop/LDA-Analysis/’ is set as standard.
@parameter: filename, type = str
@parameter: out\_dir, type = str

\end{fulllineitems}

\index{set\_tau() (windtunnel.Timeseries method)}

\begin{fulllineitems}
\phantomsection\label{\detokenize{index:windtunnel.Timeseries.set_tau}}\pysiglinewithargsret{\sphinxbfcode{\sphinxupquote{set\_tau}}}{\emph{milliseconds}}{}
Give tau a new value

\end{fulllineitems}

\index{weighted\_component\_mean (windtunnel.Timeseries attribute)}

\begin{fulllineitems}
\phantomsection\label{\detokenize{index:windtunnel.Timeseries.weighted_component_mean}}\pysigline{\sphinxbfcode{\sphinxupquote{weighted\_component\_mean}}}
Weigh the u and v component with its transit time through the
measurement volume. This is analoguous to the processing of the raw
data in the BSA software. Transit time weighting removes a possible
bias towards higher wind velocities. Returns the weighted u and v
component means.

\end{fulllineitems}

\index{weighted\_component\_variance (windtunnel.Timeseries attribute)}

\begin{fulllineitems}
\phantomsection\label{\detokenize{index:windtunnel.Timeseries.weighted_component_variance}}\pysigline{\sphinxbfcode{\sphinxupquote{weighted\_component\_variance}}}
Weigh the u and v component with its transit time through the
measurement volume. This is analoguous to the processing of the raw
data in the BSA software. Transit time weighting removes a possible
bias towards higher wind velocities. Returns the weighted u and v
component variance.

\end{fulllineitems}

\index{wind\_direction\_mag\_less\_180() (windtunnel.Timeseries method)}

\begin{fulllineitems}
\phantomsection\label{\detokenize{index:windtunnel.Timeseries.wind_direction_mag_less_180}}\pysiglinewithargsret{\sphinxbfcode{\sphinxupquote{wind\_direction\_mag\_less\_180}}}{}{}
Return the wind direction in the range -180 to +180 degrees.

\end{fulllineitems}


\end{fulllineitems}

\index{Timeseries\_nc (class in windtunnel)}

\begin{fulllineitems}
\phantomsection\label{\detokenize{index:windtunnel.Timeseries_nc}}\pysiglinewithargsret{\sphinxbfcode{\sphinxupquote{class }}\sphinxcode{\sphinxupquote{windtunnel.}}\sphinxbfcode{\sphinxupquote{Timeseries\_nc}}}{\emph{comp\_1}, \emph{comp\_2}, \emph{x=None}, \emph{y=None}, \emph{z=None}, \emph{t\_arr\_1=None}, \emph{t\_transit\_1=None}, \emph{t\_arr\_2=None}, \emph{t\_transit\_2=None}}{}
Timeseries is a class that holds data collected by the BSA software in
non-coincidence mode using the standard BSA software output. The class can
hold die raw timeseries, the corresponding wtref, the components and 
coordinates of each measurement as well as the mean wind magnitude and the
mean wind direction. The raw timeseries can be processed by 
nondimensionalising it, adapting the scale, making it equidistant and 
masking outliers. All the information in a Timeseries object can be saved
to a txt file.
@parameter: u, type = np.array
@parameter: v, type = np.array
@parameter: x, type = float
@parameter: y, type = float
@parameter: z, type = float
@parameter: t\_arr, type = np.array
@parameter: t\_transit, type = np.array
\index{adapt\_scale() (windtunnel.Timeseries\_nc method)}

\begin{fulllineitems}
\phantomsection\label{\detokenize{index:windtunnel.Timeseries_nc.adapt_scale}}\pysiglinewithargsret{\sphinxbfcode{\sphinxupquote{adapt\_scale}}}{\emph{scale}}{}
Convert timeseries from model scale to full scale.
@parameter: scale, type = float

\end{fulllineitems}

\index{calc\_direction() (windtunnel.Timeseries\_nc method)}

\begin{fulllineitems}
\phantomsection\label{\detokenize{index:windtunnel.Timeseries_nc.calc_direction}}\pysiglinewithargsret{\sphinxbfcode{\sphinxupquote{calc\_direction}}}{}{}
Calculate wind direction from components.

\end{fulllineitems}

\index{calc\_magnitude() (windtunnel.Timeseries\_nc method)}

\begin{fulllineitems}
\phantomsection\label{\detokenize{index:windtunnel.Timeseries_nc.calc_magnitude}}\pysiglinewithargsret{\sphinxbfcode{\sphinxupquote{calc\_magnitude}}}{}{}
Calculate wind magnitude from components.

\end{fulllineitems}

\index{equidistant() (windtunnel.Timeseries\_nc method)}

\begin{fulllineitems}
\phantomsection\label{\detokenize{index:windtunnel.Timeseries_nc.equidistant}}\pysiglinewithargsret{\sphinxbfcode{\sphinxupquote{equidistant}}}{}{}
Create equidistant time series.

\end{fulllineitems}

\index{from\_file() (windtunnel.Timeseries\_nc class method)}

\begin{fulllineitems}
\phantomsection\label{\detokenize{index:windtunnel.Timeseries_nc.from_file}}\pysiglinewithargsret{\sphinxbfcode{\sphinxupquote{classmethod }}\sphinxbfcode{\sphinxupquote{from\_file}}}{\emph{filename}}{}
Create Timeseries object from file.

\end{fulllineitems}

\index{get\_wind\_comps() (windtunnel.Timeseries\_nc method)}

\begin{fulllineitems}
\phantomsection\label{\detokenize{index:windtunnel.Timeseries_nc.get_wind_comps}}\pysiglinewithargsret{\sphinxbfcode{\sphinxupquote{get\_wind\_comps}}}{\emph{filename}}{}
Get wind components from filename.
@parameter: filename, type = str

\end{fulllineitems}

\index{get\_wtref() (windtunnel.Timeseries\_nc method)}

\begin{fulllineitems}
\phantomsection\label{\detokenize{index:windtunnel.Timeseries_nc.get_wtref}}\pysiglinewithargsret{\sphinxbfcode{\sphinxupquote{get\_wtref}}}{\emph{wtref\_path}, \emph{filename}, \emph{index=0}, \emph{vscale=1.0}}{}
Reads wtref-file selected by the time series name ‘filename’ and
scales wtref with vscale. vscale is set to 1 as standard. index
accesses only the one wtref value that is associated to the current
file.
@parameter: path, type = string
@parameter: filename, type = string
@parameter: index, type = int
@parameter: vscale, type = float

\end{fulllineitems}

\index{mask\_outliers() (windtunnel.Timeseries\_nc method)}

\begin{fulllineitems}
\phantomsection\label{\detokenize{index:windtunnel.Timeseries_nc.mask_outliers}}\pysiglinewithargsret{\sphinxbfcode{\sphinxupquote{mask\_outliers}}}{\emph{std\_mask=5.0}}{}
Mask outliers and print number of outliers. std\_mask specifies the
threshold for a value to be considered an outlier. 5 is the default
value for std\_mask.
@parameter: std\_mask, type = float

\end{fulllineitems}

\index{mean\_direction (windtunnel.Timeseries\_nc attribute)}

\begin{fulllineitems}
\phantomsection\label{\detokenize{index:windtunnel.Timeseries_nc.mean_direction}}\pysigline{\sphinxbfcode{\sphinxupquote{mean\_direction}}}
Calculate mean wind direction from components relative to the wind
tunnels axis.

\end{fulllineitems}

\index{mean\_magnitude (windtunnel.Timeseries\_nc attribute)}

\begin{fulllineitems}
\phantomsection\label{\detokenize{index:windtunnel.Timeseries_nc.mean_magnitude}}\pysigline{\sphinxbfcode{\sphinxupquote{mean\_magnitude}}}
Calculate mean wind magnitude from unweighted components.

\end{fulllineitems}

\index{nondimensionalise() (windtunnel.Timeseries\_nc method)}

\begin{fulllineitems}
\phantomsection\label{\detokenize{index:windtunnel.Timeseries_nc.nondimensionalise}}\pysiglinewithargsret{\sphinxbfcode{\sphinxupquote{nondimensionalise}}}{}{}
Nondimensionalise the data. wtref is set to 1 if no wtref is
speciefied.

\end{fulllineitems}

\index{pair\_components() (windtunnel.Timeseries\_nc method)}

\begin{fulllineitems}
\phantomsection\label{\detokenize{index:windtunnel.Timeseries_nc.pair_components}}\pysiglinewithargsret{\sphinxbfcode{\sphinxupquote{pair\_components}}}{\emph{atol=1}}{}
Pair components in comp\_1 and comp\_2 using atol as absolute
tolerance to match a pair of measurements. atol is set to 1 as default,
its unit is {[}ms{]}.
@parameter: atol, type = float or int

\end{fulllineitems}

\index{save2file() (windtunnel.Timeseries\_nc method)}

\begin{fulllineitems}
\phantomsection\label{\detokenize{index:windtunnel.Timeseries_nc.save2file}}\pysiglinewithargsret{\sphinxbfcode{\sphinxupquote{save2file}}}{\emph{filename}, \emph{out\_dir=None}}{}
Save data from Timeseries object to txt file. filename must include
‘.txt’ ending. If no out\_dir directory is provided ‘./’ is set as
standard.
@parameter: filename, type = str
@parameter: out\_dir, type = str

\end{fulllineitems}

\index{weighted\_component\_mean (windtunnel.Timeseries\_nc attribute)}

\begin{fulllineitems}
\phantomsection\label{\detokenize{index:windtunnel.Timeseries_nc.weighted_component_mean}}\pysigline{\sphinxbfcode{\sphinxupquote{weighted\_component\_mean}}}
Weigh the u and v component with its transit time through the
measurement volume. This is analoguous to the processing of the raw
data in the BSA software. Transit time weighting removes a possible
bias towards higher wind velocities. Returns the weighted u and v
component means.

\end{fulllineitems}

\index{weighted\_component\_variance (windtunnel.Timeseries\_nc attribute)}

\begin{fulllineitems}
\phantomsection\label{\detokenize{index:windtunnel.Timeseries_nc.weighted_component_variance}}\pysigline{\sphinxbfcode{\sphinxupquote{weighted\_component\_variance}}}
Weigh the u and v component with its transit time through the
measurement volume. This is analoguous to the processing of the raw
data in the BSA software. Transit time weighting removes a possible
bias towards higher wind velocities. Returns the weighted u and v
component variance.

\end{fulllineitems}


\end{fulllineitems}

\index{adapt\_scale() (in module windtunnel)}

\begin{fulllineitems}
\phantomsection\label{\detokenize{index:windtunnel.adapt_scale}}\pysiglinewithargsret{\sphinxcode{\sphinxupquote{windtunnel.}}\sphinxbfcode{\sphinxupquote{adapt\_scale}}}{\emph{x}, \emph{y}, \emph{z}, \emph{t\_arr}, \emph{scale}}{}
Convert timeseries from model scale to full scale. 
@parameter: x, type = int or float
@parameter: y, type = int or float
@parameter: z, type = int or float
@parameter: t\_arr, type = np.array
@parameter: scale, type = float

\end{fulllineitems}

\index{calc\_acorr() (in module windtunnel)}

\begin{fulllineitems}
\phantomsection\label{\detokenize{index:windtunnel.calc_acorr}}\pysiglinewithargsret{\sphinxcode{\sphinxupquote{windtunnel.}}\sphinxbfcode{\sphinxupquote{calc\_acorr}}}{\emph{timeseries}, \emph{maxlags}}{}
Full autocorrelation of time series for lags up to maxlags.
@parameter timeseries: np.array or list
@parameter maxlags: int

\end{fulllineitems}

\index{calc\_alpha() (in module windtunnel)}

\begin{fulllineitems}
\phantomsection\label{\detokenize{index:windtunnel.calc_alpha}}\pysiglinewithargsret{\sphinxcode{\sphinxupquote{windtunnel.}}\sphinxbfcode{\sphinxupquote{calc\_alpha}}}{\emph{u\_mean}, \emph{heights}, \emph{d0=0.0}, \emph{sfc\_height=120.0}, \emph{BL\_height=600.0}}{}
Estimate the power law exponent alpha.
@parameter: u\_mean, type = list or np.array
@parameter: heights, type = list or np.array
@parameter: d0, type = float
@parameter: sfc\_height, type = float
@parameter: BL\_height, type = float

\end{fulllineitems}

\index{calc\_autocorr() (in module windtunnel)}

\begin{fulllineitems}
\phantomsection\label{\detokenize{index:windtunnel.calc_autocorr}}\pysiglinewithargsret{\sphinxcode{\sphinxupquote{windtunnel.}}\sphinxbfcode{\sphinxupquote{calc\_autocorr}}}{\emph{timeseries}, \emph{lag=1}}{}
Autocorrelation of time series with lag.
@parameter tiemseries: np.array or list
@parameter lag: int

\end{fulllineitems}

\index{calc\_exceedance\_prob() (in module windtunnel)}

\begin{fulllineitems}
\phantomsection\label{\detokenize{index:windtunnel.calc_exceedance_prob}}\pysiglinewithargsret{\sphinxcode{\sphinxupquote{windtunnel.}}\sphinxbfcode{\sphinxupquote{calc\_exceedance\_prob}}}{\emph{data}, \emph{threshold}}{}
Calculates exceedance probability of threshold in data. Returns 
threshold and exceedance probability in percent.
@parameter data: 
@parameter threshold: int

\end{fulllineitems}

\index{calc\_intervalmean() (in module windtunnel)}

\begin{fulllineitems}
\phantomsection\label{\detokenize{index:windtunnel.calc_intervalmean}}\pysiglinewithargsret{\sphinxcode{\sphinxupquote{windtunnel.}}\sphinxbfcode{\sphinxupquote{calc\_intervalmean}}}{\emph{indata}, \emph{intervals}, \emph{DD=False}}{}
Calculates interval means of indata. If DD is set to True the means are 
calculated for circular quantities. Returns a dictionary with 
intervals as keys. If intervals has length 1 the function returns an array.
@parameter: indata, type = any
@parameter: intervals, type = list
@parameter: DD, type = boolean

\end{fulllineitems}

\index{calc\_lux\_data() (in module windtunnel)}

\begin{fulllineitems}
\phantomsection\label{\detokenize{index:windtunnel.calc_lux_data}}\pysiglinewithargsret{\sphinxcode{\sphinxupquote{windtunnel.}}\sphinxbfcode{\sphinxupquote{calc\_lux\_data}}}{\emph{dt}, \emph{u\_comp}}{}
Calculates the integral length scale according to R. Fischer (2011) 
from an equidistant time series of the u component using time step dt.
@parameter: t\_eq, type = int or float
@parameter: u\_comp, type = np.array or list

\end{fulllineitems}

\index{calc\_lux\_data\_wght() (in module windtunnel)}

\begin{fulllineitems}
\phantomsection\label{\detokenize{index:windtunnel.calc_lux_data_wght}}\pysiglinewithargsret{\sphinxcode{\sphinxupquote{windtunnel.}}\sphinxbfcode{\sphinxupquote{calc\_lux\_data\_wght}}}{\emph{transit\_time}, \emph{dt}, \emph{u\_comp}}{}
Calculates the integral length scale according to R. Fischer (2011) 
from an equidistant time series of the u component using time step dt.
@parameter: t\_eq, type = int or float
@parameter: u\_comp, type = np.array or list

\end{fulllineitems}

\index{calc\_normalization\_params() (in module windtunnel)}

\begin{fulllineitems}
\phantomsection\label{\detokenize{index:windtunnel.calc_normalization_params}}\pysiglinewithargsret{\sphinxcode{\sphinxupquote{windtunnel.}}\sphinxbfcode{\sphinxupquote{calc\_normalization\_params}}}{\emph{freqs}, \emph{transform}, \emph{t}, \emph{height}, \emph{mean\_x}, \emph{sdev\_x}, \emph{num\_data\_points}}{}
Calculate the normalized Fourier transform and frequency for the 
Fourier transform of x
Warning: A previous code version normalized segments, while this version 
normalizes the entire data set at once. The previous version also included 
a smoothing algorithm, which has been omitted for simplicity.
@parameter: freqs, type = list or np.array
@parameter: transform, type = list or np.array - this is the non-normalized Fourier transform
@parameter: t, type = float - this is time
@parameter: height, type = float - z in the Timeseries object
@parameter: mean\_x, type = float - the mean of the parameter of the Fourier transform F(x)
@parameter: sdev\_x, type = float - the standard deviation of x
@parameter: num\_data\_points, type = int - the number of elements in x before the transform was found

\end{fulllineitems}

\index{calc\_ref\_spectra() (in module windtunnel)}

\begin{fulllineitems}
\phantomsection\label{\detokenize{index:windtunnel.calc_ref_spectra}}\pysiglinewithargsret{\sphinxcode{\sphinxupquote{windtunnel.}}\sphinxbfcode{\sphinxupquote{calc\_ref\_spectra}}}{\emph{reduced\_freq}, \emph{a}, \emph{b}, \emph{c}, \emph{d}, \emph{e}}{}
Calculate dimensionless reference spectra. ???
@parameter: reduced\_freq, type = ???
@parameter: a, type = ???
@parameter: b, type = ???
@parameter: c, type = ???
@parameter: d, type = ???
@parameter: e, type = ???

\end{fulllineitems}

\index{calc\_spectra() (in module windtunnel)}

\begin{fulllineitems}
\phantomsection\label{\detokenize{index:windtunnel.calc_spectra}}\pysiglinewithargsret{\sphinxcode{\sphinxupquote{windtunnel.}}\sphinxbfcode{\sphinxupquote{calc\_spectra}}}{\emph{u\_comp}, \emph{v\_comp}, \emph{t\_eq}, \emph{height}}{}
Calculate dimensionless energy density spectra from an equidistant 
time series.
@parameter: u\_comp, type = np.array or list
@parameter: v\_comp, type = np.array or list
@parameter: t\_eq, type = np.array or list

\end{fulllineitems}

\index{calc\_stats() (in module windtunnel)}

\begin{fulllineitems}
\phantomsection\label{\detokenize{index:windtunnel.calc_stats}}\pysiglinewithargsret{\sphinxcode{\sphinxupquote{windtunnel.}}\sphinxbfcode{\sphinxupquote{calc\_stats}}}{\emph{sets}, \emph{DD=False}}{}
Returns mean, standard deviation and variance of data in sets. If DD is 
true then the circular equivalents are calculated. TO BE USED WITH CAUTION
@parameter sets: iterable set of data
@parameter DD: boolean

\end{fulllineitems}

\index{calc\_turb\_data() (in module windtunnel)}

\begin{fulllineitems}
\phantomsection\label{\detokenize{index:windtunnel.calc_turb_data}}\pysiglinewithargsret{\sphinxcode{\sphinxupquote{windtunnel.}}\sphinxbfcode{\sphinxupquote{calc\_turb\_data}}}{\emph{u\_comp}, \emph{v\_comp}}{}
Calculate turbulence intensity and turbulent fluxes from equidistant
times series of u and v components.
@parameter: u\_comp: np.array or list
@parameter: v\_comp: np.array or list

\end{fulllineitems}

\index{calc\_turb\_data\_wght() (in module windtunnel)}

\begin{fulllineitems}
\phantomsection\label{\detokenize{index:windtunnel.calc_turb_data_wght}}\pysiglinewithargsret{\sphinxcode{\sphinxupquote{windtunnel.}}\sphinxbfcode{\sphinxupquote{calc\_turb\_data\_wght}}}{\emph{transit\_time}, \emph{u\_comp}, \emph{v\_comp}}{}
Calculate turbulence intensity and turbulent fluxes from equidistant
times series of u and v components using transit time weighted statistics.
@parameter: transit\_time. type = np.array
@parameter: u\_compy type = np.array
@parameter: v\_comp, type = np.array

\end{fulllineitems}

\index{calc\_wind\_stats() (in module windtunnel)}

\begin{fulllineitems}
\phantomsection\label{\detokenize{index:windtunnel.calc_wind_stats}}\pysiglinewithargsret{\sphinxcode{\sphinxupquote{windtunnel.}}\sphinxbfcode{\sphinxupquote{calc\_wind\_stats}}}{\emph{u\_comp}, \emph{v\_comp}, \emph{wdir=0.0}}{}
Calculate wind data from equidistant times series of u and 
v components. wdir is a reference wind direction.
@parameter: u\_comp: np.array or list
@parameter: v\_comp: np.array or list
@parameter: wdir: int

\end{fulllineitems}

\index{calc\_wind\_stats\_wght() (in module windtunnel)}

\begin{fulllineitems}
\phantomsection\label{\detokenize{index:windtunnel.calc_wind_stats_wght}}\pysiglinewithargsret{\sphinxcode{\sphinxupquote{windtunnel.}}\sphinxbfcode{\sphinxupquote{calc\_wind\_stats\_wght}}}{\emph{transit\_time}, \emph{u\_comp}, \emph{v\_comp}, \emph{wdir=0.0}}{}
Calculate wind data from equidistant times series of u and 
v components. wdir is a reference wind direction.
@parameter: transit\_time, type = np.array
@parameter: u\_comp, type = np.array
@parameter: v\_comp, type = np.array
@parameter: wdir: int

\end{fulllineitems}

\index{calc\_z0() (in module windtunnel)}

\begin{fulllineitems}
\phantomsection\label{\detokenize{index:windtunnel.calc_z0}}\pysiglinewithargsret{\sphinxcode{\sphinxupquote{windtunnel.}}\sphinxbfcode{\sphinxupquote{calc\_z0}}}{\emph{u\_mean}, \emph{heights}, \emph{d0=0.0}, \emph{sfc\_height=120.0}, \emph{BL\_height=600.0}}{}
Estimate the roughness length z0.
@parameter: u\_mean, type = list or np.array
@parameter: heights, type = list or np.array
@parameter: d0, type = float
@parameter: sfc\_height, type = float
@parameter: BL\_height, type = float

\end{fulllineitems}

\index{check\_directory() (in module windtunnel)}

\begin{fulllineitems}
\phantomsection\label{\detokenize{index:windtunnel.check_directory}}\pysiglinewithargsret{\sphinxcode{\sphinxupquote{windtunnel.}}\sphinxbfcode{\sphinxupquote{check\_directory}}}{\emph{directory}}{}
Checks if directory exists. If directory doesn’t exist, it is created.
@parameter: directory, type = string

\end{fulllineitems}

\index{convergence\_test\_1() (in module windtunnel)}

\begin{fulllineitems}
\phantomsection\label{\detokenize{index:windtunnel.convergence_test_1}}\pysiglinewithargsret{\sphinxcode{\sphinxupquote{windtunnel.}}\sphinxbfcode{\sphinxupquote{convergence\_test\_1}}}{\emph{data}, \emph{blocksize=100}}{}
Conducts a block-wise convergence test on non circular data using 
blocksize for the size of each increment. Returns a dictionary block\_data.
Each entry is named after its respective interval. blocksize’s default 
value is 100.
@parameter: data, type = np.array or list
@parameter: blocksize, type = int or float

\end{fulllineitems}

\index{convergence\_test\_2() (in module windtunnel)}

\begin{fulllineitems}
\phantomsection\label{\detokenize{index:windtunnel.convergence_test_2}}\pysiglinewithargsret{\sphinxcode{\sphinxupquote{windtunnel.}}\sphinxbfcode{\sphinxupquote{convergence\_test\_2}}}{\emph{data}, \emph{interval=100}, \emph{blocksize=100}}{}
Conducts a block-wise convergence test on non circular data using 
blocksize for the size of each increment between intervals. Returns a 
dictionary block\_data. Each entry is named after its respective interval.
blocksize’s and interval’s default values are 100.
@parameter: data, type = np.array or list
@parameter: interval, type = int
@parameter: blocksize, type = int

\end{fulllineitems}

\index{count\_nan\_chunks() (in module windtunnel)}

\begin{fulllineitems}
\phantomsection\label{\detokenize{index:windtunnel.count_nan_chunks}}\pysiglinewithargsret{\sphinxcode{\sphinxupquote{windtunnel.}}\sphinxbfcode{\sphinxupquote{count\_nan\_chunks}}}{\emph{data}}{}
Counts chunks of NaNs in data. Returns the size of each chunk and
the overall number of chunks.
@parameter: data, type = np.array or string

\end{fulllineitems}

\index{equ\_dist\_ts() (in module windtunnel)}

\begin{fulllineitems}
\phantomsection\label{\detokenize{index:windtunnel.equ_dist_ts}}\pysiglinewithargsret{\sphinxcode{\sphinxupquote{windtunnel.}}\sphinxbfcode{\sphinxupquote{equ\_dist\_ts}}}{\emph{arrival\_time}, \emph{eq\_dist\_array}, \emph{data}}{}
Create a time series with constant time steps. The nearest point of the 
original time series is used for the corresponding time of the equi-distant
time series.
@parameter: arrival\_time, type = np.array
@parameter: eq\_dist\_array, type = np.array
@parameter: data, type = np.array

\end{fulllineitems}

\index{equidistant() (in module windtunnel)}

\begin{fulllineitems}
\phantomsection\label{\detokenize{index:windtunnel.equidistant}}\pysiglinewithargsret{\sphinxcode{\sphinxupquote{windtunnel.}}\sphinxbfcode{\sphinxupquote{equidistant}}}{\emph{u}, \emph{v}, \emph{t\_arr}}{}
Create equidistant time series.
@parameter: u, type = np.array
@parameter: v, type = np.array
@parameter: t\_arr, type = np.array or list

\end{fulllineitems}

\index{find\_block() (in module windtunnel)}

\begin{fulllineitems}
\phantomsection\label{\detokenize{index:windtunnel.find_block}}\pysiglinewithargsret{\sphinxcode{\sphinxupquote{windtunnel.}}\sphinxbfcode{\sphinxupquote{find\_block}}}{\emph{indata}, \emph{length}, \emph{tolerance}}{}
Finds block of size length in indata. Tolerance allows some leeway.
Returns array.
@parameter: indata, type = np.array (1D)
@parameter: length, type = int
@parameter: tolerance, type = int

\end{fulllineitems}

\index{find\_nearest() (in module windtunnel)}

\begin{fulllineitems}
\phantomsection\label{\detokenize{index:windtunnel.find_nearest}}\pysiglinewithargsret{\sphinxcode{\sphinxupquote{windtunnel.}}\sphinxbfcode{\sphinxupquote{find\_nearest}}}{\emph{array}, \emph{value}}{}
Finds nearest element of array to value.
@parameter: array, np.array
@parameter: value, int or float

\end{fulllineitems}

\index{from\_file() (in module windtunnel)}

\begin{fulllineitems}
\phantomsection\label{\detokenize{index:windtunnel.from_file}}\pysiglinewithargsret{\sphinxcode{\sphinxupquote{windtunnel.}}\sphinxbfcode{\sphinxupquote{from\_file}}}{\emph{path}, \emph{filename}}{}
Create array from timeseries in path + file.
@parameter: path, string
@parameter: filename, string

\end{fulllineitems}

\index{get\_files() (in module windtunnel)}

\begin{fulllineitems}
\phantomsection\label{\detokenize{index:windtunnel.get_files}}\pysiglinewithargsret{\sphinxcode{\sphinxupquote{windtunnel.}}\sphinxbfcode{\sphinxupquote{get\_files}}}{\emph{path}, \emph{filename}}{}
Finds files with filename in path as specified. Filename supports the
Unix shell-style wildcards ({\color{red}\bfseries{}*},?,{[}seq{]},{[}!seq{]})
@parameter: path, type = string
@parameter: filename, type = string

\end{fulllineitems}

\index{get\_lux\_referencedata() (in module windtunnel)}

\begin{fulllineitems}
\phantomsection\label{\detokenize{index:windtunnel.get_lux_referencedata}}\pysiglinewithargsret{\sphinxcode{\sphinxupquote{windtunnel.}}\sphinxbfcode{\sphinxupquote{get\_lux\_referencedata}}}{\emph{ref\_path=None}}{}
Reads and returns reference data for the integral length scale (Lux).
This function takes no parameters.

\end{fulllineitems}

\index{get\_pdf\_max() (in module windtunnel)}

\begin{fulllineitems}
\phantomsection\label{\detokenize{index:windtunnel.get_pdf_max}}\pysiglinewithargsret{\sphinxcode{\sphinxupquote{windtunnel.}}\sphinxbfcode{\sphinxupquote{get\_pdf\_max}}}{\emph{data}}{}
Finds maximum of the probability distribution of data.
@parameter data: np.array

\end{fulllineitems}

\index{get\_percentiles() (in module windtunnel)}

\begin{fulllineitems}
\phantomsection\label{\detokenize{index:windtunnel.get_percentiles}}\pysiglinewithargsret{\sphinxcode{\sphinxupquote{windtunnel.}}\sphinxbfcode{\sphinxupquote{get\_percentiles}}}{\emph{data\_dict}, \emph{percentile\_list}}{}
Get percentiles from each entry in data\_dict specified in
percentile\_list. Returns a dictionary with the results.
@parameter: data\_dict, type = dict
@parameter: percentile\_list, type = list

\end{fulllineitems}

\index{get\_reference\_spectra() (in module windtunnel)}

\begin{fulllineitems}
\phantomsection\label{\detokenize{index:windtunnel.get_reference_spectra}}\pysiglinewithargsret{\sphinxcode{\sphinxupquote{windtunnel.}}\sphinxbfcode{\sphinxupquote{get\_reference\_spectra}}}{\emph{height}, \emph{ref\_path=None}}{}
Get referemce spectra from pre-defined location.

\end{fulllineitems}

\index{get\_turb\_referencedata() (in module windtunnel)}

\begin{fulllineitems}
\phantomsection\label{\detokenize{index:windtunnel.get_turb_referencedata}}\pysiglinewithargsret{\sphinxcode{\sphinxupquote{windtunnel.}}\sphinxbfcode{\sphinxupquote{get\_turb\_referencedata}}}{\emph{component}, \emph{ref\_path=None}}{}
Reads and returns the VDI reference data for the turbulence intensity of
component.
@parameter: component, type = string

\end{fulllineitems}

\index{get\_wind\_comps() (in module windtunnel)}

\begin{fulllineitems}
\phantomsection\label{\detokenize{index:windtunnel.get_wind_comps}}\pysiglinewithargsret{\sphinxcode{\sphinxupquote{windtunnel.}}\sphinxbfcode{\sphinxupquote{get\_wind\_comps}}}{\emph{path}, \emph{filename}}{}
Get wind components from filename.
@parameter: filename, type = str

\end{fulllineitems}

\index{get\_wtref() (in module windtunnel)}

\begin{fulllineitems}
\phantomsection\label{\detokenize{index:windtunnel.get_wtref}}\pysiglinewithargsret{\sphinxcode{\sphinxupquote{windtunnel.}}\sphinxbfcode{\sphinxupquote{get\_wtref}}}{\emph{wtref\_path}, \emph{filename}, \emph{index=0}, \emph{vscale=1.0}}{}
Reads wtref-file selected by the time series name ‘filename’ and
scales wtref with vscale. vscale is set to 1 as standard. index 
accesses only the one wtref value that is associated to the current
file.
@parameter: path, type = string
@parameter: filename, type = string
@parameter: index, type = int
@parameter: vscale, type = float

\end{fulllineitems}

\index{mask\_outliers() (in module windtunnel)}

\begin{fulllineitems}
\phantomsection\label{\detokenize{index:windtunnel.mask_outliers}}\pysiglinewithargsret{\sphinxcode{\sphinxupquote{windtunnel.}}\sphinxbfcode{\sphinxupquote{mask\_outliers}}}{\emph{u}, \emph{v}, \emph{std\_mask=5.0}}{}
Mask outliers and print number of outliers. std\_mask specifies the
threshold for a value to be considered an outlier. 5 is the default 
value for std\_mask.
@parameter: u, type = np.array
@parameter: v, type = np.array
@parameter: std\_mask, type = float

\end{fulllineitems}

\index{mask\_outliers\_wght() (in module windtunnel)}

\begin{fulllineitems}
\phantomsection\label{\detokenize{index:windtunnel.mask_outliers_wght}}\pysiglinewithargsret{\sphinxcode{\sphinxupquote{windtunnel.}}\sphinxbfcode{\sphinxupquote{mask\_outliers\_wght}}}{\emph{transit\_time}, \emph{u}, \emph{v}, \emph{std\_mask=5.0}}{}
Mask outliers and print number of outliers. std\_mask specifies the
threshold for a value to be considered an outlier. 5 is the default 
value for std\_mask. This function usues time transit time weighted 
statistics.
@parameter: u, type = np.array
@parameter: v, type = np.array
@parameter: std\_mask, type = float

\end{fulllineitems}

\index{nondimensionalise() (in module windtunnel)}

\begin{fulllineitems}
\phantomsection\label{\detokenize{index:windtunnel.nondimensionalise}}\pysiglinewithargsret{\sphinxcode{\sphinxupquote{windtunnel.}}\sphinxbfcode{\sphinxupquote{nondimensionalise}}}{\emph{u}, \emph{v}, \emph{wtref=None}}{}
Nondimensionalise the data. wtref is set to 1 if no wtref is 
specified.
@parameter: u, type = np.array
@parameter: v, type = np.array
@parameter: wtref, type = int or float

\end{fulllineitems}

\index{plot\_DWD\_windrose() (in module windtunnel)}

\begin{fulllineitems}
\phantomsection\label{\detokenize{index:windtunnel.plot_DWD_windrose}}\pysiglinewithargsret{\sphinxcode{\sphinxupquote{windtunnel.}}\sphinxbfcode{\sphinxupquote{plot\_DWD\_windrose}}}{\emph{inFF}, \emph{inDD}}{}
Plots windrose according to DWD classes of 1 m/s for velocity data and
30 degree classes for directional data. The representation of the windrose 
in this function is less detailed than in plotwindrose().
@parameter inFF: np.array
@parameter inDD: np.array

\end{fulllineitems}

\index{plot\_JTFA\_STFT() (in module windtunnel)}

\begin{fulllineitems}
\phantomsection\label{\detokenize{index:windtunnel.plot_JTFA_STFT}}\pysiglinewithargsret{\sphinxcode{\sphinxupquote{windtunnel.}}\sphinxbfcode{\sphinxupquote{plot\_JTFA\_STFT}}}{\emph{u1}, \emph{v1}, \emph{t\_eq}, \emph{height}, \emph{second\_comp='v'}, \emph{window\_length=3500}, \emph{fixed\_limits=(None}, \emph{None)}, \emph{ymax=None}}{}
Plots the joint time frequency analysis using a short-time Fourier
transform smoothed and raw for both wind components in one figure. Returns
the figure. To change overlap, 
@parameter: u1: array of u-component perturbations
@parameter: v1: array of second-component perturbations
@parameter: t\_eq: as defined by Timeseries
@parameter: height: z as defined by Timeseries
@parameter: second\_comp, type = string: the name of the second measured
\begin{quote}

wind component
\end{quote}

@parameter: window\_length, type = int: window length in ms

\end{fulllineitems}

\index{plot\_Re\_independence() (in module windtunnel)}

\begin{fulllineitems}
\phantomsection\label{\detokenize{index:windtunnel.plot_Re_independence}}\pysiglinewithargsret{\sphinxcode{\sphinxupquote{windtunnel.}}\sphinxbfcode{\sphinxupquote{plot\_Re\_independence}}}{\emph{data}, \emph{wtref}, \emph{yerr=0}, \emph{ax=None}, \emph{**kwargs}}{}
Plots the results for a Reynolds Number Independence test from a non-
dimensionalised timeseries. yerr specifies the uncertainty. Its default 
value is 0.
@parameter: data, type = np.array or list
@parameter: wtref, type = np.array or list
@parameter: yerr, type = int or float
@parameter: ax: axis passed to function
@parameter: {\color{red}\bfseries{}**}kwargs: additional keyword arguments passed to plt.plot()

\end{fulllineitems}

\index{plot\_boxplots() (in module windtunnel)}

\begin{fulllineitems}
\phantomsection\label{\detokenize{index:windtunnel.plot_boxplots}}\pysiglinewithargsret{\sphinxcode{\sphinxupquote{windtunnel.}}\sphinxbfcode{\sphinxupquote{plot\_boxplots}}}{\emph{data\_dict}, \emph{ylabel=None}, \emph{**kwargs}}{}
Plot statistics of concentration measurements in boxplots. Expects
input from PointConcentration class.
@parameters: data, type = dict
@parameters: ylabel, type = string
@parameter ax: axis passed to function
@parameter {\color{red}\bfseries{}**}kwargs : additional keyword arguments passed to plt.boxplot()

\end{fulllineitems}

\index{plot\_convergence() (in module windtunnel)}

\begin{fulllineitems}
\phantomsection\label{\detokenize{index:windtunnel.plot_convergence}}\pysiglinewithargsret{\sphinxcode{\sphinxupquote{windtunnel.}}\sphinxbfcode{\sphinxupquote{plot\_convergence}}}{\emph{data\_dict}, \emph{ncols=3}, \emph{**kwargs}}{}
Plots results of convergence tests performed on any number of 
quantities in one plot. ncols specifies the number of columns desired in
the output plot. {\color{red}\bfseries{}**}kwargs contains any parameters to be passed to 
plot\_convergence\_test, such as wtref, ref\_length and scale. See doc\_string
of plot\_convergence\_test for more details.
@parameter: data\_dict, type = dictionary
@parameter: ncols, type = int
@parameter: {\color{red}\bfseries{}**}kwargs keyword arguments passed to plot\_convergence\_test

\end{fulllineitems}

\index{plot\_convergence\_test() (in module windtunnel)}

\begin{fulllineitems}
\phantomsection\label{\detokenize{index:windtunnel.plot_convergence_test}}\pysiglinewithargsret{\sphinxcode{\sphinxupquote{windtunnel.}}\sphinxbfcode{\sphinxupquote{plot\_convergence\_test}}}{\emph{data}, \emph{wtref=1}, \emph{ref\_length=1}, \emph{scale=1}, \emph{ylabel=''}, \emph{ax=None}, \emph{**kwargs}}{}
Plots results of convergence tests  from data. This is a very limited 
function and is only intended to give a brief overview of the convergence
rest results using dictionaries as input objects. wtref, ref\_length and 
scale are used to determine a dimensionless time unit on the x-axis. 
Default values for each are 1.
@parameter: data\_dict, type = dictionary
@parameter: wtref, type = float or int
@parameter: ref\_length, type = float or int
@parameter: scale, type = float or int
@parameter: ylabel, type = string
@parameter: ax: axis passed to function

\end{fulllineitems}

\index{plot\_fluxes() (in module windtunnel)}

\begin{fulllineitems}
\phantomsection\label{\detokenize{index:windtunnel.plot_fluxes}}\pysiglinewithargsret{\sphinxcode{\sphinxupquote{windtunnel.}}\sphinxbfcode{\sphinxupquote{plot\_fluxes}}}{\emph{data}, \emph{heights}, \emph{yerr=0}, \emph{component='v'}, \emph{lat=False}, \emph{ax=None}, \emph{**kwargs}}{}
Plots fluxes from data for their respective height with a 10\% range of
the low point mean. yerr specifies the uncertainty. Its default value is 0.
WARNING: Data must be made dimensionless before plotting! If lat is True 
then a lateral profile is created.
@parameter: data, type = list or np.array
@parameter: height, type = list or np.array
@parameter: yerr, type = int or float
@parameter: component, type = string
@parameter: lat, type = boolean
@parameter ax: axis passed to function
@parameter {\color{red}\bfseries{}**}kwargs : additional keyword arguments passed to plt.plot()

\end{fulllineitems}

\index{plot\_fluxes\_log() (in module windtunnel)}

\begin{fulllineitems}
\phantomsection\label{\detokenize{index:windtunnel.plot_fluxes_log}}\pysiglinewithargsret{\sphinxcode{\sphinxupquote{windtunnel.}}\sphinxbfcode{\sphinxupquote{plot\_fluxes\_log}}}{\emph{data}, \emph{heights}, \emph{yerr=0}, \emph{component='v'}, \emph{ax=None}, \emph{**kwargs}}{}
Plots fluxes from data for their respective height on a log scale with
a 10\% range of the low point mean. yerr specifies the uncertainty. Its 
default value is 0. WARNING: Data must be made dimensionless before 
plotting!
@parameter: data, type = list or np.array
@parameter: height, type = list or np.array
@parameter: yerr, type = int or float
@parameter: component, type = string
@parameter ax: axis passed to function
@parameter {\color{red}\bfseries{}**}kwargs : additional keyword arguments passed to plt.plot()

\end{fulllineitems}

\index{plot\_hist() (in module windtunnel)}

\begin{fulllineitems}
\phantomsection\label{\detokenize{index:windtunnel.plot_hist}}\pysiglinewithargsret{\sphinxcode{\sphinxupquote{windtunnel.}}\sphinxbfcode{\sphinxupquote{plot\_hist}}}{\emph{data}, \emph{ax=None}, \emph{**kwargs}}{}
Creates a scatter plot of x and y.
@parameter: data, type = list or np.array
@parameter ax: axis passed to function
@parameter {\color{red}\bfseries{}**}kwargs : additional keyword arguments passed to plt.plot()

\end{fulllineitems}

\index{plot\_lux() (in module windtunnel)}

\begin{fulllineitems}
\phantomsection\label{\detokenize{index:windtunnel.plot_lux}}\pysiglinewithargsret{\sphinxcode{\sphinxupquote{windtunnel.}}\sphinxbfcode{\sphinxupquote{plot\_lux}}}{\emph{Lux}, \emph{heights}, \emph{err=0}, \emph{lat=False}, \emph{ref\_path=None}, \emph{ax=None}, \emph{**kwargs}}{}
Plots Lux data on a double logarithmic scale with reference data. yerr
specifies the uncertainty. Its default value is 0. If lat
is True then a lateral profile, without a loglog scale, is created.
@parameter: Lux, type = list or np.array
@parameter: heights, type = list or np.array
@parameter: err, type = int or float
@parameter: lat, type = boolean
@parameter: ref\_path = string
@parameter ax: axis passed to function
@parameter {\color{red}\bfseries{}**}kwargs : additional keyword arguments passed to plt.plot()

\end{fulllineitems}

\index{plot\_perturbation\_rose() (in module windtunnel)}

\begin{fulllineitems}
\phantomsection\label{\detokenize{index:windtunnel.plot_perturbation_rose}}\pysiglinewithargsret{\sphinxcode{\sphinxupquote{windtunnel.}}\sphinxbfcode{\sphinxupquote{plot\_perturbation\_rose}}}{\emph{u1}, \emph{v1}, \emph{total\_mag}, \emph{total\_direction}, \emph{bar\_divider=3000}, \emph{second\_comp='v'}}{}
Plots a detailed wind rose using only the perturbation component of
the wind. Number of bars depends on bar\_divider and length of u1.
@parameter: u1: array of u-component perturbations
@parameter: v1: array of second-component perturbations
@parameter: total\_mag: array containing magnitude of wind (not perturbation)
@parameter: total\_direction: array containing direction of wind (not perturbation)
@parameter: bar\_divider: inversely proportional to number of bars to be plotted
@parameter: second\_comp, type = string: the name of the second measured
\begin{quote}

wind component
\end{quote}

\end{fulllineitems}

\index{plot\_rose() (in module windtunnel)}

\begin{fulllineitems}
\phantomsection\label{\detokenize{index:windtunnel.plot_rose}}\pysiglinewithargsret{\sphinxcode{\sphinxupquote{windtunnel.}}\sphinxbfcode{\sphinxupquote{plot\_rose}}}{\emph{inFF}, \emph{inDD}, \emph{ff\_steps}, \emph{dd\_range}}{}
Plots windrose according to user specified input from ff\_steps and
dd\_Range.
@parameter: inFF, type = np.array
@parameter: inDD, type = np.array
@parameter: ff\_steps, type = list or np.array
@parameter: dd\_range, type = int or float

\end{fulllineitems}

\index{plot\_scatter() (in module windtunnel)}

\begin{fulllineitems}
\phantomsection\label{\detokenize{index:windtunnel.plot_scatter}}\pysiglinewithargsret{\sphinxcode{\sphinxupquote{windtunnel.}}\sphinxbfcode{\sphinxupquote{plot\_scatter}}}{\emph{x}, \emph{y}, \emph{std\_mask=5.0}, \emph{ax=None}, \emph{**kwargs}}{}
Creates a scatter plot of x and y. All outliers outside of 5 STDs of the
components mean value are coloured in orange.
@parameter: x, type = list or np.array
@parameter: y, type = list or np.array
@parameter: std\_mask, float
@parameter ax: axis passed to function
@parameter {\color{red}\bfseries{}**}kwargs : additional keyword arguments passed to plt.scatter()

\end{fulllineitems}

\index{plot\_spectra() (in module windtunnel)}

\begin{fulllineitems}
\phantomsection\label{\detokenize{index:windtunnel.plot_spectra}}\pysiglinewithargsret{\sphinxcode{\sphinxupquote{windtunnel.}}\sphinxbfcode{\sphinxupquote{plot\_spectra}}}{\emph{f\_sm}, \emph{S\_uu\_sm}, \emph{S\_vv\_sm}, \emph{S\_uv\_sm}, \emph{u\_aliasing}, \emph{v\_aliasing}, \emph{uv\_aliasing}, \emph{wind\_comps}, \emph{height}, \emph{ref\_path=None}, \emph{ax=None}, \emph{**kwargs}}{}
Plots spectra using INPUT with reference data.
@parameter: ???
@parameter: ref\_path, type = string
@parameter ax: axis passed to function
@parameter {\color{red}\bfseries{}**}kwargs : additional keyword arguments passed to plt.plot()

\end{fulllineitems}

\index{plot\_stdevs() (in module windtunnel)}

\begin{fulllineitems}
\phantomsection\label{\detokenize{index:windtunnel.plot_stdevs}}\pysiglinewithargsret{\sphinxcode{\sphinxupquote{windtunnel.}}\sphinxbfcode{\sphinxupquote{plot\_stdevs}}}{\emph{u\_unmasked}, \emph{t\_eq}, \emph{tau}, \emph{comp='u'}}{}
This method plots the spread of an array based on how many standard 
deviations each point is from the mean over each tau-long time period
@parameter: the array to be analysed
@parameter: the times corresponding to the array to be analysied (ms)
@parameter: the characteristic time scale (ms)

\end{fulllineitems}

\index{plot\_turb\_int() (in module windtunnel)}

\begin{fulllineitems}
\phantomsection\label{\detokenize{index:windtunnel.plot_turb_int}}\pysiglinewithargsret{\sphinxcode{\sphinxupquote{windtunnel.}}\sphinxbfcode{\sphinxupquote{plot\_turb\_int}}}{\emph{data}, \emph{heights}, \emph{yerr=0}, \emph{component='I\_u'}, \emph{lat=False}, \emph{ref\_path=None}, \emph{ax=None}, \emph{**kwargs}}{}
Plots turbulence intensities from data with VDI reference data for 
their respective height. yerr specifies the uncertainty. Its default value
is 0. If lat is True then a lateral profile is created.
@parameter: data, type = list or np.array
@parameter: heights, type = list or np.array
@parameter: yerr, type = int or float
@parameter: component, type = string
@parameter: lat, type = boolean
@parameter: ref\_path, type = string
@parameter: ax, axis passed to function    
@parameter {\color{red}\bfseries{}**}kwargs : additional keyword arguments passed to plt.plot()

\end{fulllineitems}

\index{plot\_wind\_dir\_hist() (in module windtunnel)}

\begin{fulllineitems}
\phantomsection\label{\detokenize{index:windtunnel.plot_wind_dir_hist}}\pysiglinewithargsret{\sphinxcode{\sphinxupquote{windtunnel.}}\sphinxbfcode{\sphinxupquote{plot\_wind\_dir\_hist}}}{\emph{data}, \emph{heights}}{}
Simple wind direction histogram plot
@parameter: Timeseries

\end{fulllineitems}

\index{plot\_winddata() (in module windtunnel)}

\begin{fulllineitems}
\phantomsection\label{\detokenize{index:windtunnel.plot_winddata}}\pysiglinewithargsret{\sphinxcode{\sphinxupquote{windtunnel.}}\sphinxbfcode{\sphinxupquote{plot\_winddata}}}{\emph{mean\_magnitude}, \emph{u\_mean}, \emph{v\_mean}, \emph{heights}, \emph{yerr=0}, \emph{lat=False}, \emph{ax=None}, \emph{**kwargs}}{}
Plots wind components and wind magnitude for their respective height.
yerr specifies the uncertainty. Its default value is 0. If lat is True then
a lateral profile is created.
@parameter: mean\_magnitude, type = list or np.array
@parameter: u\_mean, type = list or np.array
@parameter: v\_mean, type = list or np.array
@parameter: heights, type = list or np.array
@parameter: yerr, type = int or float
@parameter: lat, type = boolean
@parameter ax: axis passed to function
@parameter {\color{red}\bfseries{}**}kwargs : additional keyword arguments passed to plt.plot()

\end{fulllineitems}

\index{plot\_winddata\_log() (in module windtunnel)}

\begin{fulllineitems}
\phantomsection\label{\detokenize{index:windtunnel.plot_winddata_log}}\pysiglinewithargsret{\sphinxcode{\sphinxupquote{windtunnel.}}\sphinxbfcode{\sphinxupquote{plot\_winddata\_log}}}{\emph{mean\_magnitude}, \emph{u\_mean}, \emph{v\_mean}, \emph{heights}, \emph{yerr=0}, \emph{ax=None}, \emph{**kwargs}}{}
Plots wind components and wind magnitude for their respective height on
a log scale. yerr specifies the uncertainty. Its default value is 0.
@parameter: mean\_magnitude, type = list or np.array
@parameter: u\_mean, type = list or np.array
@parameter: v\_mean, type = list or np.array
@parameter: heights, type = list or np.array
@parameter: yerr, type = int or float
@parameter: ax, axis passed to function
@parameter {\color{red}\bfseries{}**}kwargs : additional keyword arguments passed to plt.plot()

\end{fulllineitems}

\index{plotcdfs() (in module windtunnel)}

\begin{fulllineitems}
\phantomsection\label{\detokenize{index:windtunnel.plotcdfs}}\pysiglinewithargsret{\sphinxcode{\sphinxupquote{windtunnel.}}\sphinxbfcode{\sphinxupquote{plotcdfs}}}{\emph{sets}, \emph{lablist}, \emph{ax=None}, \emph{**kwargs}}{}
Plots CDFs of data in sets using the respective labels from lablist
@parameter sets: iterable set of data
@parameter lablist: list of strings
@parameter ax: axis passed to function
@parameter {\color{red}\bfseries{}**}kwargs : additional keyword arguments passed to plt.plot()

\end{fulllineitems}

\index{plotpdfs() (in module windtunnel)}

\begin{fulllineitems}
\phantomsection\label{\detokenize{index:windtunnel.plotpdfs}}\pysiglinewithargsret{\sphinxcode{\sphinxupquote{windtunnel.}}\sphinxbfcode{\sphinxupquote{plotpdfs}}}{\emph{sets}, \emph{lablist}, \emph{ax=None}, \emph{**kwargs}}{}
Plots PDFs of data in sets using the respective labels from lablist.
@parameter sets: iterable set of data
@parameter lablist: list of strings
@parameter ax: axis passed to function
@parameter {\color{red}\bfseries{}**}kwargs : additional keyword arguments passed to plt.plot()

\end{fulllineitems}

\index{plotpdfs\_err() (in module windtunnel)}

\begin{fulllineitems}
\phantomsection\label{\detokenize{index:windtunnel.plotpdfs_err}}\pysiglinewithargsret{\sphinxcode{\sphinxupquote{windtunnel.}}\sphinxbfcode{\sphinxupquote{plotpdfs\_err}}}{\emph{sets}, \emph{lablist}, \emph{error}, \emph{ax=None}, \emph{**kwargs}}{}
Plots PDFs of data in sets using the respective labels from lablist with
a given margin of error.
@parameter sets: iterable set of data
@parameter lablist: list of strings
@parameter error: int or float
@parameter ax: axis passed to function
@parameter {\color{red}\bfseries{}**}kwargs : additional keyword arguments passed to plt.plot()

\end{fulllineitems}

\index{plotwindrose() (in module windtunnel)}

\begin{fulllineitems}
\phantomsection\label{\detokenize{index:windtunnel.plotwindrose}}\pysiglinewithargsret{\sphinxcode{\sphinxupquote{windtunnel.}}\sphinxbfcode{\sphinxupquote{plotwindrose}}}{\emph{inFF}, \emph{inDD}, \emph{num\_bars=10}, \emph{ax=None}, \emph{left\_legend=False}}{}
Plots windrose with dynamic velocity classes of each 10\% percentile and
10 degree classes for directional data. The representation of the windrose 
in this function is more detailed than in plot\_DWD\_windrose().
@parameter inFF: np.array
@parameter inDD: np.array
@parameter num\_bars: how many segments the degree range should be broken
\begin{quote}

into
\end{quote}

@parameter ax: pyplot axes object, must be polar
@left\_legend: if true, the legend is positioned to the left of the plot
\begin{quote}

instead of the right
\end{quote}

\end{fulllineitems}

\index{power\_law() (in module windtunnel)}

\begin{fulllineitems}
\phantomsection\label{\detokenize{index:windtunnel.power_law}}\pysiglinewithargsret{\sphinxcode{\sphinxupquote{windtunnel.}}\sphinxbfcode{\sphinxupquote{power\_law}}}{\emph{u\_comp}, \emph{height}, \emph{u\_ref}, \emph{z\_ref}, \emph{alpha}, \emph{d0=0}}{}
Estimate power law profile.
@parameter: u\_comp, type = int or float
@parameter: height, type = int or float
@parameter: u\_ref, type = int or float
@parameter: z\_ref, type = int or float
@parameter: alpha, type = int or float
@parameter: d0, type = int or float

\end{fulllineitems}

\index{transit\_time\_weighted\_flux() (in module windtunnel)}

\begin{fulllineitems}
\phantomsection\label{\detokenize{index:windtunnel.transit_time_weighted_flux}}\pysiglinewithargsret{\sphinxcode{\sphinxupquote{windtunnel.}}\sphinxbfcode{\sphinxupquote{transit\_time\_weighted\_flux}}}{\emph{transit\_time}, \emph{component\_1}, \emph{component\_2}}{}
Calculate mean flux using transit time weighted statistics. Transit
time weighting removes a possible bias towards higher wind velocities.
Returns a mean weighted flux.
@parameter: transit\_time, type = np.arrray({[}{]})
@parameter: component\_1,  type = np.arrray({[}{]})
@parameter: component\_2,  type = np.arrray({[}{]})

\end{fulllineitems}

\index{transit\_time\_weighted\_mean() (in module windtunnel)}

\begin{fulllineitems}
\phantomsection\label{\detokenize{index:windtunnel.transit_time_weighted_mean}}\pysiglinewithargsret{\sphinxcode{\sphinxupquote{windtunnel.}}\sphinxbfcode{\sphinxupquote{transit\_time\_weighted\_mean}}}{\emph{transit\_time}, \emph{component}}{}
Weigh the flow component with its transit time through the
measurement volume. This is analoguous to the processing of the raw
data in the BSA software. Transit time weighting removes a possible
bias towards higher wind velocities. Returns the weighted component mean.
@parameter: transit\_time, type = np.arrray({[}{]})
@parameter: component,  type = np.arrray({[}{]})

\end{fulllineitems}

\index{transit\_time\_weighted\_var() (in module windtunnel)}

\begin{fulllineitems}
\phantomsection\label{\detokenize{index:windtunnel.transit_time_weighted_var}}\pysiglinewithargsret{\sphinxcode{\sphinxupquote{windtunnel.}}\sphinxbfcode{\sphinxupquote{transit\_time\_weighted\_var}}}{\emph{transit\_time}, \emph{component}}{}
Weigh the u and v component with its transit time through the
measurement volume. This is analoguous to the processing of the raw
data in the BSA software. Transit time weighting removes a possible
bias towards higher wind velocities. Returns the weighted u and v
component variance.
@parameter: transit\_time, type = np.arrray({[}{]})
@parameter: component,  type = np.arrray({[}{]})

\end{fulllineitems}

\index{trunc\_at() (in module windtunnel)}

\begin{fulllineitems}
\phantomsection\label{\detokenize{index:windtunnel.trunc_at}}\pysiglinewithargsret{\sphinxcode{\sphinxupquote{windtunnel.}}\sphinxbfcode{\sphinxupquote{trunc\_at}}}{\emph{string}, \emph{delimiter}, \emph{n=3}}{}
Returns string truncated at the n’th (3rd by default) occurrence of the
delimiter.

\end{fulllineitems}



\chapter{Indices and tables}
\label{\detokenize{index:indices-and-tables}}\begin{itemize}
\item {} 
\DUrole{xref,std,std-ref}{genindex}

\item {} 
\DUrole{xref,std,std-ref}{modindex}

\item {} 
\DUrole{xref,std,std-ref}{search}

\end{itemize}


\renewcommand{\indexname}{Python Module Index}
\begin{sphinxtheindex}
\def\bigletter#1{{\Large\sffamily#1}\nopagebreak\vspace{1mm}}
\bigletter{w}
\item {\sphinxstyleindexentry{windtunnel}}\sphinxstyleindexpageref{index:\detokenize{module-windtunnel}}
\end{sphinxtheindex}

\renewcommand{\indexname}{Index}
\printindex
\end{document}